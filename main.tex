\documentclass[12pt]{article}
\usepackage[utf8]{inputenc}
\setlength{\marginparwidth}{2.8cm}
\usepackage{geometry, amsmath, subfig, mathtools, amssymb, 
            float, appendix, todonotes, ebgaramond, placeins,
            paralist, titlesec, afterpage, placeins,
            mathtools, mathrsfs, todonotes, needspace, siunitx,
            booktabs}
% Make numbers in math mode in Garamond too (which should not affect Greek letters
\DeclareSymbolFont{numbers}{T1}{EBGaramond-LF}{m}{n}
\SetSymbolFont{numbers}{bold}{T1}{EBGaramond-LF}{bx}{n}
\DeclareMathSymbol{0}\mathalpha{numbers}{"30}
\DeclareMathSymbol{1}\mathalpha{numbers}{"31}
\DeclareMathSymbol{2}\mathalpha{numbers}{"32}
\DeclareMathSymbol{3}\mathalpha{numbers}{"33}
\DeclareMathSymbol{4}\mathalpha{numbers}{"34}
\DeclareMathSymbol{5}\mathalpha{numbers}{"35}
\DeclareMathSymbol{6}\mathalpha{numbers}{"36}
\DeclareMathSymbol{7}\mathalpha{numbers}{"37}
\DeclareMathSymbol{8}\mathalpha{numbers}{"38}
\DeclareMathSymbol{9}\mathalpha{numbers}{"39}
            
\usepackage[longnamesfirst]{natbib} % The option writes out all authors names at first citation
\setlength{\bibsep}{0pt plus 0.3ex}
\usepackage{yhmath} % To get widehat

% https://en.wikibooks.org/wiki/LaTeX/Hyperlinks
\usepackage[bookmarks=true, colorlinks=true, citecolor=blue]{hyperref}

\usepackage{pdflscape}
\geometry{letterpaper, margin=1.25in}
\titlespacing*{\section}{0pt}{1ex}{0.5ex}

\title{A title}
\date{Last updated: \today}
 
\author{Jesper Bojeryd \& co-authors}

\newcommand{\LL}{\mathscr{L}}
\newcommand{\EE}{\mathbb{E}}
\newcommand{\PP}{\mathbb{P}}
\newcommand{\Var}{\mathbb{V}\text{ar}}
\newcommand{\Cov}{\mathbb{C}\text{ov}}

% These, with the siunitx package, make it possible to align numbers in tables
%   by the decimal point. From https://tex.stackexchange.com/questions/411352/dcolumn-and-aligning-string-values
%\newcolumntype{d}[1]{D{.}{.}{#1}}
\sisetup{
    table-align-text-before = false,
    table-align-text-after  = false,
    input-open-uncertainty  = ,
    input-close-uncertainty = ,
}
 
\usepackage{lipsum} % Only to work with template
 
\begin{document}
 
\begingroup\makeatletter
\renewcommand{\thefootnote}{\fnsymbol{footnote}}
  \centering 
  \LARGE\@title\footnote{For helpful comments and discussions we want to thank people. All mistakes are our own.} \\[0.2em]
  \large\@author\footnote{University of California, Los Angeles. Corresponding author: \url{jesperbojeryd@ucla.edu}.}\\[0.5em]
  \@date
  \\[1em]\par
\makeatother\endgroup

\setcounter{footnote}{0}% Reset footnote counter

\begin{abstract}
    \noindent An abstract that summarizes the paper nicely
\end{abstract}

\noindent
I prefer to write the introduction as an unnumbered section. \lipsum[1]

\section{The first section with a number}
The first sentence. And I learned stuff from \cite{paper2022}. \citep[Example of self citation:][followed by more text.]{bojeryd1991}

\lipsum[2-3]

\subsection{Examples of table with nicely aligned decimal points}
This is an example of a table, or, it is at Table \ref{tab:baseline}.

% Each S[table-format=...] has to be tailored to fit numbers, parentheses and stars.
% To illustrate, one standard error is centered and not aligned with decimal point.
\begin{table}[t]\centering
    \caption{Some regressions}\label{tab:baseline}
    \begin{tabular}{lS[table-format=-1.3)]S[table-format=-1.5{**}]S[table-format=-1.2{***}]S[table-format=-2.1)e2, parse-numbers=false]}
    \toprule
                        & {(1)}     & {(2)}      & {(3)}    & {(4)}              \\ \midrule
    Post $\times$ Small Non-Fin.
                        & -0.026    & -0.00042** & -0.046   & 6.6 \times 10^8    \\
                        & (0.051)   & (0.055)    & (0.047)  & (0.082)            \\
    log Assets          &           & -0.19***   & -0.20*** & 21.1\times 10^{10} \\
                        &           & (0.059)    & (0.041)  &                    \\
    Other controls      & {No}      & {No}       & {Yes}    & {No}               \\
    \bottomrule
    \end{tabular}
\end{table}

\lipsum[5-6]




\bibliographystyle{econ}
\bibliography{references}


\newpage\clearpage
\appendix 
\setcounter{figure}{0} \renewcommand{\thefigure}{A\arabic{figure}}
\setcounter{table}{0} \renewcommand{\thetable}{A\arabic{table}}

\section*{Appendix}
Stuff that goes into an appendix

\lipsum[7]

% This table uses sisetup instead
\begin{table}[h]
    \centering
    \sisetup{table-format=-1.3, table-space-text-post={***}}
    \begin{tabular}{lSScSS}
    & \multicolumn{2}{c}{$\log y_1$} && 
      \multicolumn{2}{c}{$\log y_2$} \\
     \cmidrule{2-3}\cmidrule{5-6}
            & {(1)}     & {(2)}     && {(3)}     & {(4)}     \\ \midrule
    $\beta$
            & -0.214*** & -0.194*** &&  0.002    & -0.019    \\
            & (0.035)   & (0.029)   && (0.014)   & (0.014)   \\
    Other variable
            &  0.211*** &  0.222*** &&  0.135*** &  0.135*** \\
            & (0.035)   & (0.034)   && (0.032)   & (0.032)   \\ \midrule
    Observations 
            &  {$11{,}308$} & {$19{,}666$}  && {$50{,}248$}  & {$85{,}048$}  \\
    $R$-squared
            & {$0.330$}   & {$0.374$}   && {$0.430$}   & {$0.437$}   \\ \bottomrule
    \end{tabular}
    \caption{Caption}
    \label{tab:my_label}
\end{table}

\end{document}


